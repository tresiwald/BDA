\chapter{Evaluation}




\begin{itemize}
    \item Idealer Lucene-Score ist nicht zwingend ideal für Tagging.
    \item Mass für Generalität.
    \item Tags nach Anzahl Dokumente klassifizieren
    \item Gutes Tag
    
    \begin{longtable}{|p{4cm}| p{4cm}| p{4cm}|}
  \hline
    \textbf{Tag} & \textbf{\# Dokumente}& \textbf{\# Score}\\\hline
        \caption{Definition: Gutes Tag}
    \label{gutes-tag}
\end{longtable}
    \item Gewichtung der Dokument-Länge
    \item Disambiguation: Beispiel Jaguar (Tier oder Automarke?)
    
\end{itemize}


\section{Anforderungen}

\section{Experten-Feedback}

\begin{itemize}
    \item Nutzen für Praxis
    
\end{itemize}

\subsection{Michael Kaufmann}

\subsection{Kevin Stadelmann}

\section{Diskussion}

Michael oder wir

\section{Case Space}

Qualitative Analyse

\section{Implikationen für die Praxis}

kritisch, brauchbar?

\subsection{Bedeutung der Schlüsselwörter}