%\usepackage{glossaries}
\usepackage[acronym,noredefwarn]{glossaries}
\makenoidxglossaries

\newglossaryentry{ikc-core}
{
    name=ikc-core,
    description={Bestehender Software-Prototyp aus dem IKC-Projekt}
}

\newglossaryentry{Intuitive Knowledge Connectivity}
{
    name=Intuitive Knowledge Connectivity,
    description={Forschungsprojekt der Hochschule Luzern - Informatik, welches sich mit plattformübergreifenden Wissennetzwerken beschäftigt},
}

\newglossaryentry{IKC}
{
    name=IKC,
    description={Forschungsprojekt der Hochschule Luzern - Informatik, welches sich mit plattformübergreifenden Wissennetzwerken beschäftigt},
}


\newglossaryentry{Dropbox}
{
    name=Dropbox,
    description={Cloud-Service zum Austausch von Daten}
    \cite{dropbox}
}

\newglossaryentry{BDA}
{
    name=BDA,
    description={aaa}
}


\newglossaryentry{Evernote}
{
    name=Evernote,
    description={Web-basierte Applikation zur Erfassung, Speicherung und Synchronisation von Text, Bildern oder Videos}
    \cite{evernote}
}

\newglossaryentry{cloudbasiert}
{
    name=cloudbasiert,
    description={Cloud-Computing: Nutzung von IT-Infrastrukturen und -Dienstleistungen, die nicht vor Ort auf lokalen Rechnern vorgehalten, sondern als Dienst gemietet werden und auf die über ein Netzwerk (z.B. das Internet) zugegriffen wird} \cite{duden.de}
}

\newglossaryentry{PAWI}
{
    name=PAWI,
    description={Informatik-Projekt der Studierenden an der Hochschule Luzern.}
}

\newglossaryentry{Responsive Design}
{
    name=Responsive Design,
    description={Webseiten sollen nicht nur im Layout / Design nicht nur flexibler werden, sondern sich dem Medium anpassen}
    \cite[S.8]{responsive-webdesign}
}

\newglossaryentry{Usability}
{
    name=Usability,
    description={Usability eines Produktes ist das Ausmaß, in dem es von einem bestimmten Benutzer verwendet werden kann, um bestimmte Ziele in einem bestimmten Kontext \textbf{effektiv}, \textbf{effizient} und \textbf{zufriedenstellend} zu erreichen}
    \cite{usab}
}

\newglossaryentry{User Experience}
{
    name=User Experience,
    description={User Experience erweitert das Konzept der \gls{Usability} um ästhetische und emotionale Faktoren}
    \cite{ux}
}

\newglossaryentry{User Interface}
{
    name=User Interface,
    description={Das User Interface ist für den Benutzer die (grafische) Schnittstelle zum Prototypen}
}

\newglossaryentry{mobile first}
{
    name=mobile first,
    description={Der Prototyp wird in erster Linie für die Darstellung auf mobilen Endgeräten (Tables und Smartphones) entwickelt}
}

\newglossaryentry{Node}
{
    name=Node,
    description={Element innerhalb eines Netzwerk.}
}

\newglossaryentry{Link}
{
    name=Link,
    description={verbindet zwei \gls{Node}[s] in einem Netzwerk}
}

\newglossaryentry{View}
{
    name=View,
    description={Benutzerbasierte Visualisierung eines (Teil-)Netzwerks}
}

\newglossaryentry{Callback}
{
    name=Callback,
    description={Methodenreferenz als Parameter. So kann eine einseitige Kopplung realisiert werden}
}

\newglossaryentry{Array}
{
    name=Array,
    description={Datenstruktur, welche eine Sammlung von Elementen enthält}
}

\newglossaryentry{Scala}
{
    name=Scala,
    description={JVM-basierte Programmiersprachen mit modernen Sprachkonstrukten}
}

\newglossaryentry{Canvas}
{
    name=Canvas,
    description={\gls{HTML}-Element, welches zum Zeichnen benutzt werden kann}
}

\newglossaryentry{HTML}
{
    name=HTML,
    description={Hypertext Markup Language, welche die Struktur einer Webseite bildet}
}

\newglossaryentry{CSS}
{
    name=CSS,
    description={Cascading Stylesheets, ist eine Computersprache für die Gestaltung von \gls{HTML}-Elementen}
}

\newglossaryentry{Drag'n'Drop}
{
    name=Drag'n'Drop,
    description={Methode zur Bedienung einer Benutzeroberfläche, wobei ein Element mit der Maus oder dem Finger 'gepackt' und anschliessend 'losgelassen' wird}
}

\newglossaryentry{Event}
{
    name=Event,
    description={\gls{HTML} oder \gls{Javascript}-Ereignis, welche beispielsweise durch eine Benutzerinteraktion ausgelöst wird. Diese Ereignisse können im \gls{Javascript} abgefangen und so darauf entsprechend reagiert werden}
}

\newglossaryentry{Javascript}
{
    name=Javascript,
    description={Skriptsprache für Webprogrammierung}
}

\newglossaryentry{MoSCoW-System}
{
    name=MoSCoW-System,
    description={Technik zur Priorisierung von Anforderungen}
    \cite{moscow:hardvard}
}

\newglossaryentry{Tags}
{
    name=Tags,
    description={Tags welche im \gls{ikc-core} benutzt werden können, um Nodes zu gruppieren}
}

\newglossaryentry{Props}
{
    name=Props,
    description={React-spezifische Eigenschaften von Komponenten (vgl. \hyperref[props]{React: State-Schnittstelle}}
}

\newglossaryentry{State}
{
    name=State,
    description={React-spezifische Zustände von Komponenten (vgl. \hyperref[props]{React: Props-Schnittstelle}}
}

\newglossaryentry{Framework}
{
    name=Framework,
    description={Programmatische Erweiterung oder Zusatzbibliothek}
}

\newglossaryentry{Toolbox}
{
    name=Toolbox,
    description={Werkzeugplalette, welche beispielsweise Zu\-satz\-funk\-tio\-na\-li\-tät\-en bietet}
}

\newglossaryentry{Netzwerk}
{
    name=Netzwerk,
    description={Netzwerk als Sammlung von \gls{Node}[s] und \gls{Link}[s]}
}

\newglossaryentry{Datenquelle}
{
    name=Datenquelle,
    description={tbd}
    \cite{dropbox}
}

\newglossaryentry{Tag Recommondation}
{
    name=Tag Recommondation,
    description={tbd}
    \cite{dropbox}
}

\newglossaryentry{n-gram}
{
    name=n-gram,
    description={tbd}
    \cite{dropbox}
}

\newglossaryentry{Tag}
{
    name=Tag,
    description={tbd}
    \cite{dropbox}
}

\newglossaryentry{Keyword}
{
    name=Keyword,
    description={tbd}
    \cite{dropbox}
}

\newglossaryentry{Keyword Extraction}
{
    name=Keyword Extraction,
    description={tbd}
    \cite{dropbox}
}

\newglossaryentry{SFTP}
{
    name=SFTP,
    description={tbd}
    \cite{dropbox}
}

\newglossaryentry{Dokku}
{
    name=Dokku,
    description={tbd}
    \cite{dropbox}
}

\newglossaryentry{PoC}
{
    name=PoC,
    description={tbd}
}
\newglossaryentry{Natural Language Processing}
{
    name=Natural Language Processing,
    description={tbd}
}
\newglossaryentry{Textanalyse}
{
    name=Textanalyse,
    description={tbd}
}
\newglossaryentry{Volltextsuche}
{
    name=Volltextsuche,
    description={tbd}
}
\newglossaryentry{Linguistik}
{
    name=Linguistik,
    description={tbd}
}

\newglossaryentry{Konzept}
{
    name=Konzept,
    description={tbd}
}
\newglossaryentry{Machine Learning}
{
    name=Machine Learning,
    description={tbd}
}
\newglossaryentry{Artifical Intelligence}
{
    name=Artifical Intelligence,
    description={tbd}
}
\newglossaryentry{Computer Vision}
{
    name=Computer Vision,
    description={tbd}
}
\newglossaryentry{React}
{
    name=React,
    description={tbd}
}
\newglossaryentry{Stream}
{
    name=Stream,
    description={tbd}
}
\newglossaryentry{Text-Korpus}
{
    name=Text-Korpus,
    description={tbd}
}
\newglossaryentry{Ubuntu}
{
    name=Ubuntu,
    description={tbd}
}
\newglossaryentry{Docker}
{
    name=Docker,
    description={tbd}
}
\newglossaryentry{Gitlab CI}
{
    name=Gitlab CI,
    description={tbd}
}
\newglossaryentry{Typescript}
{
    name=Typescript,
    description={tbd}
}
\newglossaryentry{Token}
{
    name=Token,
    description={tbd}
}
\newglossaryentry{N-Gramm}
{
    name=N-Gramm,
    description={tbd}
}
\newglossaryentry{in-browser Datenbank}
{
    name=in-browser Datenbank,
    description={tbd}
}
\newglossaryentry{Websocket}
{
    name=Websocket,
    description={\autoref{literatur-kommunikation}}
}
\newglossaryentry{Buffer}
{
    name=Buffer,
    description={tbd}
}
\newglossaryentry{SSH}
{
    name=SSH,
    description={tbd}
}
\newglossaryentry{Extraction}
{
    name=Extraction,
    description={tbd}
}


\renewcommand{\glstextformat}[1]{\textbf{\itshape #1}}

% Beispiele


%The \Gls{latex} typesetting markup language is specially suitable 
%for documents that include \gls{maths}. 

%\acrlong{gcd}, which is abbreviated \acrshort{gcd}. This 
%process is similar to that used for the \acrfull{lcm}.