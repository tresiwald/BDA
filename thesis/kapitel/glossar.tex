\usepackage[acronym,noredefwarn,toc]{glossaries}
\makenoidxglossaries

\newglossaryentry{ikc-core}
{
    name=ikc-core,
    description={Bestehender Software-Prototyp aus dem IKC-Projekt}
}

\newglossaryentry{Intuitive Knowledge Connectivity}
{
    name=Intuitive Knowledge Connectivity,
    description={Forschungsprojekt der Hochschule Luzern - Informatik, welches sich mit plattformübergreifenden Wissennetzwerken beschäftigt},
}

\newglossaryentry{IKC}
{
    name=IKC,
    description={Forschungsprojekt der Hochschule Luzern - Informatik, welches sich mit plattformübergreifenden Wissennetzwerken beschäftigt (siehe auch \autoref{s-ikc}).},
}


\newglossaryentry{Dropbox}
{
    name=Dropbox,
    description={Cloud-Service zum Austausch von Daten}
    \cite{dropbox}
}

\newglossaryentry{BDA}
{
    name=BDA,
    description={Bachelor-Diplomarbeit der Studierenden an der Hochschule Luzern}
}


\newglossaryentry{Evernote}
{
    name=Evernote,
    description={Web-basierte Applikation zur Erfassung, Speicherung und Synchronisation von Text, Bildern oder Videos}
    \cite{evernote}
}

\newglossaryentry{cloudbasiert}
{
    name=cloudbasiert,
    description={Cloud-Computing: Nutzung von IT-Infrastrukturen und -Dienstleistungen, die nicht vor Ort auf lokalen Rechnern vorgehalten, sondern als Dienst gemietet werden und auf die über ein Netzwerk (z.B. das Internet) zugegriffen wird} \cite{duden.de}
}

\newglossaryentry{PAWI}
{
    name=PAWI,
    description={Informatik-Projekt der Studierenden an der Hochschule Luzern.}
}

\newglossaryentry{Responsive Design}
{
    name=Responsive Design,
    description={Webseiten sollen nicht nur im Layout / Design nicht nur flexibler werden, sondern sich dem Medium anpassen}
    \cite[S.8]{responsive-webdesign}
}

\newglossaryentry{Usability}
{
    name=Usability,
    description={Usability eines Produktes ist das Ausmaß, in dem es von einem bestimmten Benutzer verwendet werden kann, um bestimmte Ziele in einem bestimmten Kontext \textbf{effektiv}, \textbf{effizient} und \textbf{zufriedenstellend} zu erreichen}
    \cite{usab}
}

\newglossaryentry{User Experience}
{
    name=User Experience,
    description={User Experience erweitert das Konzept der \gls{Usability} um ästhetische und emotionale Faktoren}
    \cite{ux}
}

\newglossaryentry{User Interface}
{
    name=User Interface,
    description={Das User Interface ist für den Benutzer die (grafische) Schnittstelle zum Prototypen}
}

\newglossaryentry{mobile first}
{
    name=mobile first,
    description={Der Prototyp wird in erster Linie für die Darstellung auf mobilen Endgeräten (Tables und Smartphones) entwickelt}
}

\newglossaryentry{Node}
{
    name=Node,
    description={Element innerhalb eines Netzwerk.}
}

\newglossaryentry{Link}
{
    name=Link,
    description={verbindet zwei \gls{Node}[s] in einem Netzwerk}
}

\newglossaryentry{View}
{
    name=View,
    description={Benutzerbasierte Visualisierung eines (Teil-)Netzwerks}
}

\newglossaryentry{Callback}
{
    name=Callback,
    description={Methodenreferenz als Parameter. So kann eine einseitige Kopplung realisiert werden}
}

\newglossaryentry{Array}
{
    name=Array,
    description={Datenstruktur, welche eine Sammlung von Elementen enthält}
}

\newglossaryentry{Scala}
{
    name=Scala,
    description={JVM-basierte Programmiersprachen mit modernen Sprachkonstrukten}
}

\newglossaryentry{Canvas}
{
    name=Canvas,
    description={\gls{HTML}-Element, welches zum Zeichnen benutzt werden kann}
}

\newglossaryentry{HTML}
{
    name=HTML,
    description={Hypertext Markup Language, welche die Struktur einer Webseite bildet}
}

\newglossaryentry{CSS}
{
    name=CSS,
    description={Cascading Stylesheets, ist eine Computersprache für die Gestaltung von \gls{HTML}-Elementen}
}

\newglossaryentry{Drag'n'Drop}
{
    name=Drag'n'Drop,
    description={Methode zur Bedienung einer Benutzeroberfläche, wobei ein Element mit der Maus oder dem Finger 'gepackt' und anschliessend 'losgelassen' wird}
}

\newglossaryentry{Event}
{
    name=Event,
    description={\gls{HTML} oder \gls{Javascript}-Ereignis, welche beispielsweise durch eine Benutzerinteraktion ausgelöst wird. Diese Ereignisse können im \gls{Javascript} abgefangen und so darauf entsprechend reagiert werden}
}

\newglossaryentry{Javascript}
{
    name=Javascript,
    description={Skriptsprache für Webprogrammierung}
}

\newglossaryentry{MoSCoW-System}
{
    name=MoSCoW-System,
    description={Technik zur Priorisierung von Anforderungen}
    \cite{moscow:hardvard}
}

\newglossaryentry{Keyword Extraction}
{
    name=Keyword Extraction,
    description={Vorgang zum Auffinden von Schlüsselwörtern oder Konzepten innerhalb eines Textes (auf Basis eines Korpus.)}
}

\newglossaryentry{Tags}
{
    name=Tags,
    description={Tags welche im \gls{ikc-core} benutzt werden können, um Nodes zu gruppieren}
}

\newglossaryentry{Props}
{
    name=Props,
    description={React-spezifische Eigenschaften von Komponenten (vgl. \hyperref[props]{React: State-Schnittstelle}}
}

\newglossaryentry{State}
{
    name=State,
    description={React-spezifische Zustände von Komponenten (vgl. \hyperref[props]{React: Props-Schnittstelle}}
}

\newglossaryentry{Framework}
{
    name=Framework,
    description={Programmatische Erweiterung oder Zusatzbibliothek}
}

\newglossaryentry{Toolbox}
{
    name=Toolbox,
    description={Werkzeugplalette, welche beispielsweise Zu\-satz\-funk\-tio\-na\-li\-tät\-en bietet}
}

\newglossaryentry{Netzwerk}
{
    name=Netzwerk,
    description={Netzwerk als Sammlung von \gls{Node}[s] und \gls{Link}[s]}
}

\newglossaryentry{Datenquelle}
{
    name=Datenquelle,
    description={Orte oder Services von welchem Daten bezogen werden können.}
}

\newglossaryentry{Tag Recommondation}
{
    name=Tag Recommondation,
    description={Automatische Vorschläge von Begriffen, welcher der Kategorisierung von Inhalten dienen.}
}

\newglossaryentry{N-Gramm}
{
    name=N-Gramm,
    description={N aufeinanderfolgende Fragmente (Wörter) eines Textes bilden ein N-Gramm.}
}

\newglossaryentry{Tag}
{
    name=Tag,
    description={Ein Wort oder eine Wortkombination, welche zur Kate\-gor\-i\-sier\-ung oder Kenn\-zeich\-nung von Dingen benutzt werden kann.}
}

\newglossaryentry{Keyword}
{
    name=Keyword,
    description={Eine \gls{Keyphrase}, welche die Wortlänge eins besitzt, somit nur aus einem Wort besteht, wird als Keyword bezeichnet.}
}

\newglossaryentry{Keyphrase}
{
    name=Key\-phrase,
    description={Eine Keyphrase ist ein aus einem oder mehreren Worten bestehender Begriff mit einem für den Menschen verständlichen Sinn. Dieser soll den zugrunde liegenden Text im besten Fall kurz und prägnant zusammenfassen. Mit dieser Semantik  spielen Keyphrases eine wichtige Grundlagen für diverse Anwendungen im Bereich des Information Retrieval.}
}

\newglossaryentry{Keyphrase Extraction}
{
    name=Keyphrase Extraction,
    description={Vorgang zur Ermittlung von \gls{Keyphrase}[s] aus einem Text auf Basis eines Dokumentenkorpus.}
}

\newglossaryentry{SFTP}
{
    name=SFTP,
    description={SSH File Transfer Protocol: Für \gls{SSH} entwickeltes File Transfer Protocol, welches Verschlüsselung unterstützt.}
}

\newglossaryentry{Dokku}
{
    name=Dokku,
    description={Docker-basierte PaaS, welche die Handhabung und Entwicklung von Applikationen erleichtert.}
}

\newglossaryentry{PoC}
{
    name=PoC,
    description={Proof of Concept: Belegt die prinzipielle Durchführbarkeit eines Vorhabens.}
}

\newglossaryentry{Natural Language Processing}
{
    name=Natural Language Processing,
    description={Bereich der Informatik und speziell in der künstlichen Intelligenz, welcher an der Interaktion zwischen menschlicher Sprache und Computern interessiert ist. Text- und Sprachverständnis sind bilden die Grundlage für viele weitere Anwendungen. Wissen über die Strukturen von Text, Sätzen und Wörtern ist hier von grosser Bedeutung.}
}

\newglossaryentry{Textanalyse}
{
    name=Textanalyse,
    description={Befasst sich mit der Analyse von Texten aufgrund von Zählen, Gruppierungen und Statistik. Text wird hier wie her\-köm\-mliche Daten und ohne zusätzliches Wissen wie etwa Grammatik und Wortarten behandelt.}
}
\newglossaryentry{Volltextsuche}
{
    name=Volltextsuche,
    description={Suche über den gesamten Inhalt einer Datei oder eines Dokuments.}
}
\newglossaryentry{Linguistik}
{
    name=Linguistik,
    description={Sprachwissenschaft}
}

\newglossaryentry{Konzept}
{
    name=Konzept,
    description={Für den Menschen sinnvolle und verständliche Dinge.}
}
\newglossaryentry{Machine Learning}
{
    name=Machine Learning,
    description={Machine Learning umfasst diverse Methoden zur künstlichen Generierung von Wissen. Dies geschieht anhand von Beispielen, also aufgrund von Erfahrungswerten.}
}

\newglossaryentry{Artifical Intelligence}
{
    name=Artifical Intelligence,
    description={Gebiet der Informatik, welches sich mit dem intelligenten Automatisierung von Abläufen befasst.}
}

\newglossaryentry{Computer Vision}
{
    name=Computer Vision,
    description={Bereich der Information, welcher rechnergestützte Lösungen für Aufgaben sucht, welche sich am menschlichen Sehen orientieren.}
}

\newglossaryentry{React}
{
    name=React,
    description={Eine von Facebook entwickelte Javascript-Bibliothek zum Aufbau von Benutzeroberflächen.}
}

\newglossaryentry{Stream}
{
    name=Stream,
    description={Kontinuierlicher Datenfluss, beispielsweise vom Dateisystem oder über das Netzwerk.}
}
\newglossaryentry{Text-Korpus}
{
    name=Text-Korpus,
    description={Gesamter Textinhalt}
}
\newglossaryentry{Ubuntu}
{
    name=Ubuntu,
    description={Kostenlose, weit verbreite auf Debian basierte Linux-Dis\-tri\-bu\-tion, sowohl als Server- und Desktop-Version erhältlich.}
}
\newglossaryentry{Docker}
{
    name=Docker,
    description={System zur Isolation und Entwicklung von Anwendungen mithilfe von Betriebssystem-Visualisierungen}
}
\newglossaryentry{Gitlab CI}
{
    name=Gitlab CI,
    description={Gitlab stellt Services für Continuous Integration und -De\-ploy\-ment zur Verfügung. Dies ermöglicht autmatisierte Builds, Testing und Deployment.}
}
\newglossaryentry{Typescript}
{
    name=Typescript,
    description={Superset von Javascript, welche moderne Möglichbeiten bietet und zu Javascript kompiliert wird.}
}
\newglossaryentry{Token}
{
    name=Token,
    description={Hier ein Schlüssel, welcher zur eindeutigen Benutzeridentifikation dient.}
}

\newglossaryentry{in-browser Datenbank}
{
    name=in-browser Datenbank,
    description={Client-seitige Datenbank, welche direkt im Browser läuft und auch persistiert wird.}
}
\newglossaryentry{Websocket}
{
    name=Websocket,
    description={Siehe \autoref{literatur-kommunikation}}
}
\newglossaryentry{Buffer}
{
    name=Buffer,
    description={Ein Buffer ist ein Speicherbereich für eine Software, in welchem sie Daten zwischenspeichern kann. Hier steht es aber für eine Instanz der Buffer-Klasse von Javascript.}
}
\newglossaryentry{SSH}
{
    name=SSH,
    description={Secure Shell, ist ein Netzwerkprotokoll für verschlüsselte Verbindungen.}
}
\newglossaryentry{Extraktion}
{
    name=Extraktion,
    description={Das Suchen und Finden, je nach dem Entfernen, einer Sache.}
}
\newglossaryentry{Bibliothek}
{
    name=Bibliothek,
    description={Sammlung von Zusatzprogrammen}
}

\newglossaryentry{elasticlunr}
{
    name=elasticlunr,
    description={Auf lunr basierter, schmaler Volltextindex, optimiert für die Verwendung im Browser.}
}
\newglossaryentry{regex}
{
    name=regex,
    description={Auch regulärer Ausdruck ist eine Zeichenkette, welche eine Menge von Zeichenketten beschreibt. Dient der Filterung und der Mustererkennung.}
}

\newglossaryentry{lunr}
{
    name=lunr,
    description={Javascript Volltextindex Implementation \url{https://lunrjs.com/}}
}

\newglossaryentry{msgpack}
{
    name=msgpack,
    description={Binäres Serialisierungsformat \url{http://msgpack.org/}}
}

\newglossaryentry{lz4}
{
    name=lz4,
    description={Komprimierungsalgorithmus \url{https://github.com/lz4/lz4}}
}

\newglossaryentry{Lucene}
{
    name=Lucene,
    description={In Java implementierter Volltextservice \url{http://lucene.apache.org/core/}}
}

\newglossaryentry{gitlab}
{
    name=gitlab,
    description={Web-Applikation zur Verwaltung von Software-Projekten. Be\-inhaltet Funktionen wie ein Versionskontrollsystem, Bug-Track\-ing, CI, CD und vieles mehr.}
}

\newglossaryentry{enterpriselab}
{
    name=enterpriselab,
    description={(Anbieter von ) Server-Infrastruktur der Hochschule Luzern}
}

\newglossaryentry{Score}
{
    name=Score,
    description={Metrik, welche Vergleiche und Priorisierung ermöglicht.}
}

\newglossaryentry{YAML}
{
    name=YAML,
    description={Vereinfachte Auszeichnungssprache, angelehnt an XML.}
}

\newglossaryentry{git}
{
    name=git,
    description={Versionskontrollsystem}
}



\renewcommand{\glstextformat}[1]{\textbf{\itshape #1}}

% Beispiele


%The \Gls{latex} typesetting markup language is specially suitable 
%for documents that include \gls{maths}. 

%\acrlong{gcd}, which is abbreviated \acrshort{gcd}. This 
%process is similar to that used for the \acrfull{lcm}.