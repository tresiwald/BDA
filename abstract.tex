\begin{abstract}


%Für die Auswahl der relevanten Begriffe werden Methoden aus der Statistik, als auch aus der Sprachverarbeitung verwendet. Erstere werden genutzt, um die entsprechende Relevanz eines Begriffs zu berechnen. Die letztere dient der Reduktion potentieller Begriffe. Dabei werden unter anderem bekannte Sprachstrukturen, welche relevante Begriffe auszeichnen, berücksichtigt. 

%Der ideale, relevante Begriff zeichnet sich durch eine hohe Relevanz innerhalb des Textes, aber einer niedrigen Relevanz innerhalb des ganzen \gls{Text-Korpus} aus. Weiter darf er jedoch nicht zu spezifisch sein, da er den Text in einem gewissen Masse zusammenfassen soll.




% Wissen und Information, cooles Zitat
% http://faculty.ung.edu/kmelton/Documents/DataWisdom.pdf

%Im Forschungsprojekt \textbf{\textit{IKC}}\footnote{\gls{Intuitive Knowledge Connectivity}} wird ein Prototyp für den Umgang mit einem plattformübergreifenden Wissensnetzwerk entwickelt. Obwohl die grundlegende Datenbasis ebenfalls in Form eines Netzwerks aufgebaut ist, wählt die Benutzeroberfläche einen anderen Ansatz: Bisher handelt es sich lediglich um eine technische Konsole. Diese macht für den Benutzer zwar alle existierenden Funktionalitäten zugänglich, jedoch ist sie weder sonderlich effizient noch benutzerfreundlich. Für einen ersten Machbarkeitsnachweis war dies ausreichend. Um den Prototypen aber einem grösseren Nutzerkreis verfügbar zu machen, gibt es Optimierungsbedarf. Auch macht es aus Sicht des Projektteams Sinn, dass die Netzwerkstruktur auch für den Benutzer sichtbar ist. Das Ziel dieser Arbeit ist darum ein Prototyping einer webbasierten Visualisierung zur intuitiven Interaktion mit einem Wissensnetzwerk.
\end{abstract}
